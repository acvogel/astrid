
\documentclass[letterpaper]{article}



\usepackage{times}
\usepackage{uist}

\begin{document}

% --- Copyright notice ---
\conferenceinfo{UIST'09}{October 4-7, 2009, Victoria, British Columbia, Canada}
\CopyrightYear{2009}
\crdata{978-1-60558-745-5/09/10}

% Uncomment the following line to hide the copyright notice
% \toappear{}
% ------------------------

\bibliographystyle{plain}

%\title{Standard UIST Conference Format:\\
%       Preparing Camera-Ready Submissions}
\title{Voicify: Creating User-Generated Mobile Speech Interfaces by Demonstration}

\author{Adam Vogel \and Arda Kara}

%%
%% Note on formatting authors at different institutions, as shown below:
%% Change width arg (currently 7cm) to parbox commands as needed to
%% accommodate widest lines, taking care not to overflow the 17.8cm line width.
%% Add or delete parboxes for additional authors at different institutions. 
%% If additional authors won't fit in one row, you can add a "\\"  at the
%% end of a parbox's closing "}" to have the next parbox start a new row.
%% Be sure NOT to put any blank lines between parbox commands!
%%

%\author{
%\parbox[t]{9cm}{\centering
%	     {\em Author One}\\
%	     User Interface Laboratory\\
%             ABC Corporation\\
%	     1234 Anywhere Road\\
%	     Anytown, NY 10027 USA\\
%	     +1-212-555-1212\\
%	     one@abc.com}
%\parbox[t]{9cm}{\centering
%	     {\em Author Two}\\
%	     Universit\'{e} de XYZ\\
%	     5678 rue des Parapluies\\
%	     99099 Cr\`{e}me de Menthe, FRANCE\\
%	     +33-12-34-56-78\\
%	     deux@uvw.xyz.fr}
%}

\maketitle

\abstract
%Although voice interfaces exist for the main functions of modern mobile phones, many community-developed 
%applications lack speech support. 
We present \emph{Voicify}, a framework for voice programing by 
demonstration which enables users to build their own speech interfaces.
The user creates their own voice interface by demonstration a voice command with the corresponding
actions to take on the phone. The phone can then be put in a hands-free mode, where it
responds to previously programmed voice commands. 
Transparency is a key issue, and we discuss the development of a voice-only keyboard for
text entry paired with a custom screen reader for giving feedback to the user.
We conducted a user study which shows that
interfaces created using Voicify rival their touch counterparts.
However, our study also shows that most users find programming by demonstration to be
tedious, and that the natural language processing is too impoverished to be widely applicable.

\section{Introduction}

\section{Related Work}

\section{Programming by Demonstration}
\subsection{UI Capture}

\subsection{Demonstration Interface}

\subsection{Playback Interface}
\begin{enumerate}
\item Screen reader
\item Voice keyboard
\item Natural langage processing for command interpretation
\end{enumerate}

\section{Experimental Evaluation}
\subsection{Experimental Design}

\subsection{Results}

\subsection{Discussion}

\section{Conclusion and Future Work}



\end{document}
